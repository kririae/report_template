\documentclass{breport}
\usepackage{blindtext}

%%%%%%%%%%%%%%%%%%%%%%%%%%%%%%%%%%%%%%
%                                    %
% USER-SPECIFIED OPTIONS             %
%                                    %
%%%%%%%%%%%%%%%%%%%%%%%%%%%%%%%%%%%%%%
\title{\textbf{(A Huge Report Name)}\\ Homework \#}
\author{(Author's Name)}
\date{\today}

\begin{document}
\maketitlepage

%%%%%%%%%%%%%%%%%%%%%%%%%%%%%%%%%%%%%%
%                                    %
% DOCUMENT START                     %
%                                    %
%%%%%%%%%%%%%%%%%%%%%%%%%%%%%%%%%%%%%%

\section{Introduction}
\begin{quote}
  \textit{The following random text is purely generated by ChatGPT. I am not responsible for the accuracy.}
\end{quote}
Analytical Mechanics is a branch of classical mechanics that deals with the formulation and solving of the equations of motion for physical systems. It combines the principles of classical mechanics with the mathematical tools of calculus and differential equations. The goal of this paper is to provide an overview of the key concepts and principles of Analytical Mechanics and its applications in various fields of physics.

\section{Lagrangian Mechanics}
Lagrangian Mechanics is a powerful tool in Analytical Mechanics that provides a systematic way of describing the motion of physical systems. The key idea of Lagrangian Mechanics is to formulate the equations of motion using a function called the Lagrangian, which is defined as the difference between the kinetic and potential energy of a system. The Lagrangian is defined as follows:

\begin{equation}
L = T - V
\end{equation}

where $T$ is the kinetic energy of the system and $V$ is the potential energy of the system. The Lagrangian is used to write the equations of motion in the form of the Lagrange's equations, which are a set of differential equations that describe the motion of a system. The Lagrange's equations are given by:

\begin{equation}
\frac{d}{dt}\frac{\partial L}{\partial \dot{q_i}} - \frac{\partial L}{\partial q_i} = 0
\end{equation}

where $q_i$ is the generalized coordinate of the system and $\dot{q_i}$ is its derivative with respect to time.

\section{Hamiltonian Mechanics}
Hamiltonian Mechanics is another important tool in Analytical Mechanics that provides an alternative formulation of the equations of motion. The key idea of Hamiltonian Mechanics is to reformulate the Lagrangian in terms of a function called the Hamiltonian, which is defined as the Legendre transform of the Lagrangian. The Hamiltonian is defined as follows:

\begin{equation}
H = \sum_{i=1}^{n}p_i\dot{q_i} - L
\end{equation}

where $p_i$ is the generalized momentum conjugate to $q_i$. The Hamiltonian is used to write the equations of motion in the form of the Hamilton's equations, which are a set of first-order differential equations that describe the motion of a system. The Hamilton's equations are given by:

\begin{align}
\dot{q_i} &= \frac{\partial H}{\partial p_i} \\
\dot{p_i} &= -\frac{\partial H}{\partial q_i}
\end{align}

\section{Applications of Analytical Mechanics}
Analytical Mechanics has numerous applications in various fields of physics, including:

\begin{itemize}
\item Astrophysics: Analytical Mechanics is used to study the motion of celestial bodies, such as planets, moons, and stars.

\item Quantum Mechanics: Analytical Mechanics provides a bridge between classical mechanics and quantum mechanics and is used to study the quantum behavior of physical systems.

\item Relativity: Analytical Mechanics is used to study the effects of special and general relativity on the motion of physical systems.

\item Particle Physics: Analytical Mechanics is used to study the motion of particles, such as electrons and protons, in various physical systems.

\item Optics: Analytical Mechanics is used to study the behavior of light and electromagnetic waves in optical systems.

\end{itemize}

In conclusion, Analytical Mechanics is a fundamental branch of classical mechanics that provides a powerful framework for formulating and solving the equations of motion for physical systems. Its combination of classical mechanics and mathematics makes it a versatile tool with numerous applications in various fields of physics.

\section{Conclusion}
In this paper, we have provided an overview of the key concepts and principles of Analytical Mechanics and its applications in various fields of physics. We have introduced the Lagrangian and Hamiltonian formulations of Analytical Mechanics and shown how they can be used to study the motion of physical systems. With its combination of classical mechanics and mathematics, Analytical Mechanics remains a vital tool for advancing our understanding of the physical world.

\end{document}

\end{document}

